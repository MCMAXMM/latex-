%导言区
\documentclass{article}%book, report ,letter
\title{\heiti 我的第一篇论文}
\author{\kaishu 马胜超}
\date{\today}
\newcommand\degree{^\circ}
\usepackage{ctex}%可以使用中文的宏包
\usepackage{amsmath}
\usepackage{amssymb}
%导言区:\usepackage{graphicx}
%语法:\includegraphics[<选项>]{<文件名>}
%格式:EPS,PDF,PNG,JPEG,BMP
\usepackage{graphicx}%\emph{}
\graphicspath{{figures/},{pics/}}%图片在当前目录下的figures目录
%正文区
%公式的环境基本分为$<formula>$,\(formula)\,\begin{math}formula\end{math}
%latex通过^实现上标,通过_实现下标通过{}实现分组
%\frac{分子}{分母}
\begin{document}

    %\maketitle
    \maketitle
    \tableofcontents%生成目录
    %多行公式的排版需要导入两个包
    %\usepackage{amsmath}
    %\usepackage{amssymb}
    \begin{gather}%自带编号,能够换行
    %\times表示乘积符号
    a+b=b+a\\
    ab=ba \notag%使用\notag可以阻止编号
    \end{gather}
    %浮动体,可以实现灵活分页,给图表添加标题,并实现交叉引用
    %\begin{figrue or table}[允许位置h(here)t(top)b(bottom)p(page独立一页)]  <任意内容> %\end{figure or table}
    %\caption \label \ref
    这是我的第一张图片见figure \ref {fig-pic1}.
    \begin{figure}[htbp]
    \centering
    %\includegraphics[scale=0.8]{pic1}
    \includegraphics[height=16cm,width=10cm]{pic1}%对图片大小进行调整
     \caption{this is my figure}\label{fig-pic1}
    %\includegraphics[]{pic1}

    %\\表示换行,&表示不同的列,l,c,r分别表示右中左对齐
    %p{<宽>}-本列宽度固定,能够自动换行
    \end{figure}
    \begin{table}
    \centering
    \caption{考试成绩单}
    \begin{tabular}{|l||c|c|c|r|}
    \hline
    姓名&语文&数学&外语&备注\\
    \hline \hline
    张三&87&0.434&98.0&优秀\\
    \hline
    李四&75&64&53&补考另行通知\\
    \hline
    王二& 80&234&234&23\\
    \hline
    \end{tabular}
    \end{table}
    \section{引言}
    %//表示换行,/par表示换段落
    %在latex中多个空格视为一个空格,
    %使用\quad产生1em的空白,使用\qquad产生2em的空白
    我们的模型表现出了良好的性能
    \section{实验方法}
    ,我们的模型表现出了良好的性能
    \section{实验结果}
    \subsection{数据}
    \subsection{图表}
   ,我们的模型表现出了良好的性能
    \subsubsection{实验条件}
    \subsubsection{实验过程}
    \subsection{结果分析}
    \section{结论}
    \section{致谢}
    可以用符号语言表示为:设直角三角形$ABC$,其中$\angle C=90\degree$,则有:
    AMD发松岛枫
        发松岛枫啊多少发送到发送到速度发松岛枫
    多少分撒旦法
    %带编号的公式如下:使用\begin{equation}和\end{equation}
    \begin{equation}
    a^2=b^2+c^2
    \end{equation}
    there is a formula is $f(x)=x^2+3x$
    here is another formula $$f(x)=x^3+x^5$$
\end{document}

